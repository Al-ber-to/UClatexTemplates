\documentclass[handout,%
		aspectratio=43,% también con 196
		9pt]{beamer}
\usetheme[estilo=claro%claro/oscuro
					]{UC}
\bibliography{biblio}

\usepackage[useregional]{datetime2}

\title{Tema \emph{beamer} NO OFICIAL - Universidad de Cantabria - v0.2}
\date{\today}
\author[Autor]{Alberto Pigazo López}
\group{Área de Electrónica}
\institute[Inst]{Dpto. de Ingeniería Informática y Electrónica}
\website{https://personales.unican.es/pigazoa/}

\begin{document}

\begin{frame}
\titlepage
\end{frame}

\begin{frame}{Estructura de la presentación}
		\tableofcontents
\end{frame}

\section{Licencia}
\begin{frame}{Licencia}

\begin{itemize}
\item Los logos de la Universidad de Cantabria (UC) y la fotografía del Campus de Las Llamas están cubiertos por \emph{copyright}. Los logos de la UC se encuentran disponibles en la intranet de la UC, en \url{https://intranet.unican.es}, y la fotografía del Campus de Las Llamas en el sitio web de la Escuela de Turismo de la UC, en \url{https://euturismoaltamira.com/historia/}. Pueden usarse estos materiales en presentaciones conectadas a actividades desarrolladas en el seno de la UC.
\item El resto del tema se proporciona bajo licencia \emph{GNU General Public License v. 3} (GPLv3). Básicamente, se puede redistribuir el tema, y/o modificarlo, bajo la misma licencia. Para más información sobre la licencia GPL, ver \url{http://www.gnu.org/licenses/}.
\end{itemize}

\end{frame}

\section{Instalación}
\begin{frame}{Instalación}

El tema consiste en:
\begin{itemize}
\item 4 ficheros de estilo: \textit{beamerthemeUC.sty}, \textit{beamercolorthemeUC.sty}, \textit{beamerinnerthemeUC.sty} y \textit{beamerouterthemeUC.sty}
\item Emplea 2 logos de la UC (en la carpeta \textit{templateImages}),
\item 1 fotografía del Campus de Las Llamas (\textit{lasLlamas.jpg})
\end{itemize}
Los logos y la fotografía se encuentran en la carpeta \textit{templateImages}.

\end{frame}

\begin{frame}{Uso recomendado}{Instalación}

En local. A la carpeta con el fichero .tex de la presentación deben añadirse:

\begin{itemize}
\item los cuatro ficheros .sty,
\item la carpeta \textit{templateImages}, con los logos de la UC y la fotografía del Campus de Las Llamas,
\item hasta 5 logos adicionales (En su caso, la plantilla los usará en la diapositiva de título). Los nombres de fichero con los logos deben de ser \textit{Logo2}, \textit{Logo3} \ldots, con extensiones .png, .jpg o .pdf. (\textit{Logo2.png}) y seguir numeración secuencial.
\end{itemize}
\end{frame}

\begin{frame}{Paquetes requeridos}{Instalación}

Además de la clase Beamer, el tema requiere los siguientes paquetes: 

\begin{itemize}
\item \textit{Montserrat}: tipo de letra de la presentación.
\item \textit{tikz}: Realización de fondos del tema. Además, es una excelente herramienta para realizar esquemas y dibujos \footfullcite{manTikz}.
\item \textit{csquotes, biblatex, silence}: Citas a pie de página.
\item \textit{textpos}: Posicionamiento del texto. 
\item \textit{babel}: Soporte de idioma.
\end{itemize}

\end{frame}

\section{Utilización}
\begin{frame}[fragile]{Utilización}

Carga el tema, con dos posibles opciones:

{\scriptsize
\begin{verbatim}
	\usetheme{UC} %fondo blanco e imagen del campus en el índice
	\usetheme[estilo=oscuro]{UC} %fondo verde e imagen del campus en el índice
	\usetheme[estilo=claro,imagen=none]{UC} %fondo blanco, sin imagen en el índice
	\usetheme[imagen=./foto.jpg]{UC} %imagen alternativa en el índice
\end{verbatim}
}

Se dispone de las opciones propias de la clase beamer.

%\alert{Es necesario compilar el tema 2/3 veces para que se apliquen los cambios de la presentación correctamente !!}

\end{frame}

\begin{frame}[fragile]{Colores}{Utilización}

Los colores definidos por el tema son:
\color{UCdark}
\begin{verbatim}
\definecolor{UCdark}{rgb}{0,0.361,0.392}
\end{verbatim}
\color{UClight}
\begin{verbatim}
\definecolor{UClight}{rgb}{0,0.616,0.667}
\end{verbatim}
\color{UCorange}
\begin{verbatim}
\definecolor{UCorange}{rgb}{0.996,0.529,0.063}
\end{verbatim}
\color{UCdark}
\end{frame}

\begin{frame}{Otras consideraciones}

\begin{itemize}
\item Formato de citas al pie (\emph{footfullcite}): Puede modificarse en las líneas 65 a 80 de \textit{beamerthemeUC.sty}.
\end{itemize}

\end{frame}

\end{document}

